\pagenumbering{arabic}
\chapter{Einleitung}
\label{cha:einleitung}

\section{Motivation}
\label{sec:motivation}
\begin{spacing}{1.5}

Die Digitalisierung und der verstärkte Einsatz von Datenanalysen haben in den letzten Jahren zu erheblichen Fortschritten in vielen Bereichen geführt, darunter auch in der öffentlichen Verwaltung. Daten werden zunehmend als wertvolle Ressource betrachtet, die zur Verbesserung von Dienstleistungen und Verwaltungsprozessen genutzt werden können. Ein bedeutender Trend in diesem Zusammenhang ist die Open-Data-Bewegung, die darauf abzielt, öffentliche Daten frei zugänglich und nutzbar zu machen \cite[560]{wewer_offene_2019}. Open (Government) Data\footnote{auf Bundesebene bereits durch § 12a \acrshort{egovg} geregelt} kann Transparenz, Partizipation und Innovation fördern, indem es Interessierten aus der Öffentlichkeit ermöglicht, auf umfangreiche Datenbestände zuzugreifen und diese für verschiedenste Zwecke zu nutzen \cite[S. 11 f.]{bieker_open_2019}.

Gleichzeitig wächst das Bewusstsein für den Schutz personenbezogener Daten, insbesondere im Hinblick auf die Einhaltung gesetzlicher Datenschutzvorgaben wie der Datenschutz-Grundverordnung (\acrshort{dsgvo}). Dieser Spannungsbogen zwischen der Notwendigkeit, Daten zu nutzen und der Pflicht, diese zu schützen, stellt eine große Herausforderung für heutige Kommunen dar.

Eine vielversprechende Lösung in diesem Kontext sind synthetische Daten. Hierbei handelt es sich um künstlich generierte Daten, die auf realen Daten basieren, jedoch keine personenbezogenen Informationen enthalten. Sie sollen damit einerseits den Datenschutz gewährleisten und gleichzeitig die analytische Nutzbarkeit von Datenbeständen erhalten.

\end{spacing}
\section{Zielsetzung und Forschungsfrage}
\label{sec:ziel_und_forschungsfrage}
\begin{spacing}{1.5}

Ziel dieser Arbeit ist es, das Konzept der synthetischen Daten zu beleuchten und deren Einsatzmöglichkeiten in kommunalen Datenbeständen zu evaluieren. Dabei soll untersucht werden, inwiefern synthetische Daten eine praktikable Alternative zu traditionellen Anonymisierungsmethoden darstellen und welche Vor- und Nachteile mit ihrer Nutzung verbunden sind. Die zentrale Forschungsfrage lautet: „Inwieweit können synthetische Daten dazu beitragen, den Datenschutz zu gewährleisten und gleichzeitig die analytische Nutzbarkeit kommunaler Datenbestände sicherzustellen?“

\end{spacing}
\section{Aufbau und Struktur der Arbeit}
\label{sec:aufbau}
\begin{spacing}{1.5}

Zunächst soll die Relevanz des Themas geschildert sowie grundlegende Begriffe im Zusammenhang mit Datenschutz und synthetischen Daten erläutert werden. Es gilt zu klären, inwiefern sich das Konzept der synthetischen Daten im Vergleich zu traditionellen Anonymisierungsmethoden unterscheidet. Der Hauptteil der Arbeit besteht anschließend darin, die Anforderungen und Probleme im Kontext der Datenverarbeitung in der öffentlichen Verwaltung zu analysieren und eine konkrete Implementierung am Beispiel von Zensusdaten zu beschreiben. Hierbei wird die Problemstellung konkretisiert, ein geeignetes Synthesizer-Tool ausgewählt und die Generierung synthetischer Zensusdaten durchgeführt. Im Anschluss daran wird die Qualität und der Datenschutz der generierten Daten evaluiert und eine vergleichende Analyse mit realen Daten durchgeführt. Die Ergebnisse werden im Hinblick auf die Herausforderungen und Chancen für die öffentliche Verwaltung diskutiert sowie Grenzen und Limitationen der Arbeit aufgezeigt. Ein Fazit und ein kurzer Ausblick auf zukünftige Entwicklungen beschließen die Arbeit.

\end{spacing}