\chapter{Diskussion der Ergebnisse}
\label{cha:diskussion}

\section{Herausforderungen und Chancen für die öffentliche Verwaltung}
\begin{spacing}{1.5}

Die Nutzung synthetischer Daten in der öffentlichen Verwaltung bietet sowohl Herausforderungen als auch Chancen. Eine der größten Herausforderungen ist die Sicherstellung der Datenqualität und -validität. Die Analyse hat gezeigt, dass die synthetischen Daten zu 100 \% valide sind und eine Übereinstimmung von etwa 86 \% mit den realen Daten hinsichtlich ihrer statistischen Eigenschaften aufweisen. Diese Werte sind vielversprechend, aber es bleibt eine Restabweichung, die bei kritischen Entscheidungen berücksichtigt werden muss.

Eine weitere Herausforderung ist die Akzeptanz der synthetischen Daten innerhalb der Verwaltung. Um diese Daten als valide Grundlage für Analysen zu akzeptieren, ist Transparenz in der Datenentstehung und Vertrauen in die Qualität und Sicherheit der Daten erforderlich. Dieses Vertrauen könnte durch Schulungen und ausgewählte Pilotprojekte gefördert werden, die den Mitarbeitern der Verwaltung die Prinzipien und Methoden der Datensynthese näherbringen. Zudem ist es wichtig, regelmäßige Audits und Qualitätskontrollen durchzuführen, um die Zuverlässigkeit der synthetischen Daten zu gewährleisten.

Auf der anderen Seite bieten synthetische Daten erhebliche Chancen. Sie ermöglichen es, detaillierte Analysen durchzuführen, ohne gegen Datenschutzrichtlinien zu verstoßen. Insbesondere in Bereichen wie dem Bürger- oder Gesundheitswesen können synthetische Daten dazu beitragen, sensible Informationen zu schützen und gleichzeitig wertvolle Erkenntnisse zu gewinnen, die zur Verbesserung der öffentlichen Dienstleistungen beitragen.

\end{spacing}
\section{Grenzen und Limitationen der Arbeit}
\begin{spacing}{1.5}

Obwohl diese Arbeit wertvolle Einblicke in die Nutzung synthetischer Daten in der kommunalen Verwaltung bietet, gibt es einige Grenzen und Limitationen, die beachtet werden müssen.

Erstens stützt sich die Arbeit lediglich auf den Census Income Datensatz. Während dieser Datensatz als repräsentativ für viele demografische und sozioökonomische Merkmale gilt, kann er nicht alle möglichen Variablen und Kontexte abdecken, die in unterschiedlichen kommunalen Anwendungen relevant sein könnten. Die Ergebnisse dieser Arbeit sind somit möglicherweise nicht ohne weiteres auf andere Datensätze oder Szenarien übertragbar.

Zweitens basieren die Implementierung und Analyse auf spezifischen technischen Rahmenbedingungen und den Eigenschaften des gewählten \acrshort{ctgan}-Synthesizer-Tools von \acrshort{sdv}. Andere Tools oder technische Implementierungen könnten zu unterschiedlichen Ergebnissen führen. Die Ergebnisse und Schlussfolgerungen dieser Arbeit sind daher primär auf das verwendete Tool und die damit verbundenen Konfigurationen beschränkt.

Ein weiterer Punkt ist der eingeschränkte Fokus der Arbeit. Sie konzentriert sich hauptsächlich auf die technische Machbarkeit und die Qualität der generierten synthetischen Daten. Aspekte wie die organisatorische Implementierung, die Schulung von Mitarbeitern oder die rechtlichen Rahmenbedingungen werden nur am Rande behandelt und bleiben weitgehend unberücksichtigt. Diese Faktoren sind jedoch entscheidend für die praktische Umsetzung und Akzeptanz synthetischer Daten in der öffentlichen Verwaltung.

Nicht zuletzt wurden die ethischen und sozialen Implikationen der Nutzung synthetischer Daten in dieser Arbeit nicht eingehend untersucht. Aspekte wie die Wahrnehmung und Akzeptanz durch die Öffentlichkeit, mögliche Verzerrungen oder Diskriminierungen durch synthetische Daten und deren Auswirkungen auf die gesellschaftliche Gleichheit bleiben offen und bedürfen weiterer Forschung.

\end{spacing}