\chapter{Fazit und Ausblick}
\label{cha:fazit}
\begin{spacing}{1.5}

Die vorliegende Seminararbeit befasste sich mit der Nutzung synthetischer Daten in der öffentlichen Verwaltung, einem Thema von wachsender Bedeutung angesichts der zunehmenden Digitalisierung und des steigenden Bedarfs an datengestützten Entscheidungsprozessen.

Zusammenfassend lässt sich feststellen, dass synthetische Daten erhebliche Vorteile bieten, insbesondere in Bezug auf den Datenschutz und die Möglichkeit, detaillierte Analysen durchzuführen, ohne die Privatsphäre der Bürger zu gefährden. Die Analyse der synthetischen Daten zeigte eine hohe Validität und eine beachtliche Übereinstimmung mit realen Daten, was ihre Nützlichkeit unterstreicht. Jedoch wurden auch Grenzen und Limitationen identifiziert, die die Genauigkeit und Generalisierbarkeit der Ergebnisse beeinflussen können.

Ein zentraler Punkt der Arbeit war die Identifizierung der Herausforderungen, die mit der Nutzung synthetischer Daten verbunden sind. Dazu gehören die Sicherstellung der Datenqualität, die Akzeptanz der Daten innerhalb der Verwaltung und die technischen sowie organisatorischen Anforderungen an die Implementierung. Diese Herausforderungen erfordern eine sorgfältige Planung und die Entwicklung geeigneter Strategien, um die breite Anwendung synthetischer Daten erfolgreich zu gestalten.

Ein Blick in die Zukunft zeigt, dass synthetische Daten eine zunehmend wichtige Rolle in der öffentlichen Verwaltung spielen könnten. Zukünftige Forschung sollte sich darauf konzentrieren, die Methoden der Datensynthese weiter zu verbessern und die Anwendungsszenarien in der Praxis zu erweitern. Insbesondere detaillierte Fallstudien könnten wertvolle Einblicke in die praktischen Herausforderungen und Erfolge bei der Nutzung synthetischer Daten liefern.

\end{spacing}