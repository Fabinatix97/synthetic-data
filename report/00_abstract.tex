%\addsec{Abstract}
%\addcontentsline{toc}{section}{Abstract}
\label{sec:zusammenfassung}
\begin{center}
\large\bfseries Abstract
\end{center}
\begin{onehalfspace}

Die Sicherheit im Luftverkehr ist von entscheidender Bedeutung und die Analyse von Meldungen sicherheitsrelevanter Ereignisse (Aviation Safety Reports) spielt eine zentrale Rolle bei der Identifizierung von Unfallursachen und der Entwicklung von Präventionsstrategien. Die vorliegende Arbeit befasst sich mit der semantischen Analyse dieser Berichte mittels Text Mining, um tiefergehende Einblicke in die Häufigkeit und die Muster von Unfallursachen zu gewinnen.

Zwei Ansätze werden vorgestellt: Topic Modeling mit BERTopic und Large Language Models (LLMs). Topic Modeling ermöglicht die Identifizierung von Themen und Mustern, während LLMs auf die Klassifizierung von Unfallursachen und menschlichen Faktoren abzielen.

Die Ergebnisse zeigen, dass beide Methoden ihre Stärken und Schwächen haben. Topic Modeling eignet sich zur Themenidentifikation, erfordert aber Validierung durch Experten. LLMs zeigen gute Leistung bei der Klassifizierung einzelner Kategorien, haben aber Schwächen bei komplexen Zusammenhängen. Die Synthese beider Ansätze ermöglicht eine umfassende Analyse. Die Kombination der Methoden kann zu einer verbesserten Identifizierung und Interpretation von Unfallursachen beitragen.

Die Arbeit zeigt die Potenziale von Text Mining für die Flugsicherheitsanalyse auf, weist aber auch auf die Grenzen der Methoden und die Notwendigkeit weiterer Forschung hin.

\end{onehalfspace}

%\minisec{Abstract}
%\label{abstract}